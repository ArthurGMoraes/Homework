ARTHUR GONCALVES DE MORAES - 816479

LISTA ENCADEADA:
Classe Cell possui um inteiro "elemento" e um ponteiro para uma próxima celula;
Classe Lista possui um ponteiro para a celula inicial da lita, um para a celula final e um inteiro "tam" para seu tamanho;
addFim(), addInicio() e addPos() adicionam elementos ao fim, inicio e em uma posição desejada, respectivamente;
rmFim(), rmInicio() e rmPos() removem elementos do fim, inicio e de uma posição desejada, respectivamente;
buscar() percorre a lista e retorna true se o elemento for encontrado;
print() imprime os elementos da lista em ordem.

FILA:
Classe Cell possui um inteiro "elemento" e um ponteiro para uma próxima celula;
Classe Fila possui um ponteiro para a celula inicial da fila, um para a celula final e um inteiro "tam" para seu tamanho;
add() adiciona um elemento ao fim da fila;
rm() remove o primeiro elemento da fila;
buscar() percorre a fila e retorna true se o elemento for encontrado; 
print() imprime os elementos da fila em ordem.

PILHA:
Classe Cell possui um inteiro "elemento" e um ponteiro para uma próxima celula;
Classe Pilha possui um ponteiro para a celula do topo da pilha e um inteiro "tam" para seu tamanho;
add() adiciona um elemento ao topo da pilha;
rm() remove o elemento do topo;
buscar() percorre a pilha e retorna true se o elemento for encontrado; 
print() imprime os elementos da pilha em ordem.

MATRIZ:
Uma implementação mais simples de uma matriz uma vez que nem todos os elementos estão encadeados, porém isso não compremete a funcionaliadde.
Classe Cell possui um inteiro "elemento" e um ponteiro para a celula seguinte e a baixo;
Classe Matriz possui um ponteiro para a celula inicial da matriz, um inteiro "row" para a quantidade de linha e um inteiro "col" para a quantidade de colunas;
O construtor incializa a matriz de tamanho (row,col) com todos seus elementos = -1;
getCell() percorre o matriz até a posição informada e retorna um apontador para a celula - é utilizado pelo setValue();
setValue() permite alterar o valor de uma posição;
deleteValue() retorna o valor de uma posição para -1;
buscar() percorre a matriz e retorna true se o elemento for encontrado; 
print() imprime os elementos da matriz em ordem.
