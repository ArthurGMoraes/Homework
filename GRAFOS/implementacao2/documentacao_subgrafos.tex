
\documentclass[a4paper,12pt]{article}
\usepackage[utf8]{inputenc}
\usepackage{amsmath}
\usepackage{listings}
\usepackage{color}

\title{Documentação: Listagem de Subgrafos de um Grafo Completo}
\author{Arthur Gonçalves de Moraes}
\date{\today}

\begin{document}

\maketitle

\section{Introdução}
Este documento descreve o funcionamento de um programa em C++ que calcula e lista todos os subgrafos possíveis de um grafo completo dado o número de vértices \(n\). O programa utiliza as funções \texttt{fatorial}, \texttt{somatorio} e \texttt{listarSubgrafos} para atingir esse objetivo.

\section{Cálculo do Fatorial}
A função \texttt{fatorial} calcula o fatorial de um número \(n\). O fatorial de \(n\) é o produto de todos os números inteiros positivos menores ou iguais a \(n\) e é denotado por \(n!\).

\begin{lstlisting}[language=C++]
int fatorial(int n) {
    int fatorial = 1;
    for (int i = 1; i <= n; i++) {
        fatorial *= i;
    }
    return fatorial;
}
\end{lstlisting}

\section{Cálculo do Somatório de Subgrafos}
A função \texttt{somatorio} calcula o número total de subgrafos possíveis em um grafo completo de \(n\) vértices. Este número é obtido através de um somatório que considera todas as combinações de vértices e arestas possíveis.

\begin{lstlisting}[language=C++]
int somatorio(int n) {
    int somatorio = 0;
    for (int j = 1; j <= n; j++) {
        int tmp1 = fatorial(n) / (fatorial(j) * fatorial(n - j));
        int tmp2 = (j * (j - 1)) / 2;
        tmp2 = pow(2, tmp2);
        somatorio += (tmp1 * tmp2);
    }
    return somatorio;
}
\end{lstlisting}

\section{Listagem de Subgrafos}
A função \texttt{listarSubgrafos} utiliza os valores calculados para listar todos os subgrafos possíveis de um grafo completo com \(n\) vértices. Cada subgrafo é uma combinação de vértices e arestas.

\begin{lstlisting}[language=C++]
void listarSubgrafos(int n) {
    int total_subgrafos = somatorio(n);
    cout << "Total de subgrafos possiveis: " << total_subgrafos << endl;

    for (int i = 1; i < pow(2,n); i++) {
        vector<int> vertices;
        for (int j = 0; j < n; j++) {
            if (i & (1 << j)) {
                vertices.push_back(j + 1);
            }
        }

        int v_size = vertices.size();
        int num_arestas = (v_size * (v_size - 1)) / 2;

        for (int k = 0; k < (1 << num_arestas); k++) {
            cout << "Subgrafo: { ";
            for (int v : vertices) {
                cout << v << " ";
            }
            cout << "} - Arestas: { ";

            int edge_count = 0;
            for (int x = 0; x < v_size; x++) {
                for (int y = x + 1; y < v_size; y++) {
                    if (k & (1 << edge_count)) {
                        cout << "(" << vertices[x] << ", " << vertices[y] << ") ";
                    }
                    edge_count++;
                }
            }
            cout << "}" << endl;
        }
    }
}
\end{lstlisting}

\section{Detalhes sobre a Listagem dos Subgrafos}

\subsection{Combinações de Vértices}
A listagem dos subgrafos começa pela identificação de todas as possíveis combinações de vértices do grafo completo. Se \(n\) for o número total de vértices, então existem \(2^n - 1\) combinações não vazias de vértices. Cada uma dessas combinações representa um possível subgrafo.

No código, isso é implementado através de um loop que percorre todos os números inteiros de 1 até \(2^n - 1\). Cada número é tratado como uma representação binária onde cada bit indica a presença (\texttt{1}) ou ausência (\texttt{0}) de um vértice específico no subgrafo.

\subsection{Construção de Subgrafos com Arestas}
Para cada combinação de vértices, o programa precisa determinar todas as possíveis combinações de arestas que podem existir entre os vértices selecionados. Se o subgrafo contém \(v\) vértices, então o número máximo de arestas possíveis é dado por \( \frac{v(v-1)}{2} \), que corresponde a um grafo completo dos \(v\) vértices.

O código utiliza um loop adicional para percorrer todas as combinações de arestas possíveis. Esse loop considera todas as configurações de arestas, desde o subgrafo sem nenhuma aresta até o subgrafo que inclui todas as arestas possíveis.

\subsection{Exemplo de Listagem de Subgrafos}
Considere um grafo completo com 3 vértices, \(V = \{1, 2, 3\}\). Os subgrafos listados pelo programa incluirão:

\begin{itemize}
    \item Subgrafos com 1 vértice: \(\{1\}\), \(\{2\}\), \(\{3\}\), todos sem arestas.
    \item Subgrafos com 2 vértices: \(\{1, 2\}\), \(\{1, 3\}\), \(\{2, 3\}\), cada um com possíveis configurações de arestas: sem arestas ou com a única aresta possível.
    \item Subgrafos com 3 vértices: \(\{1, 2, 3\}\), com as configurações de arestas: sem arestas, apenas uma aresta, duas arestas ou todas as três arestas \((1, 2)\), \((1, 3)\), \((2, 3)\).
\end{itemize}

\section{Execução do Programa}
O programa principal pede ao usuário que insira o número de vértices \(n\). Em seguida, ele chama a função \texttt{listarSubgrafos} para calcular e exibir todos os subgrafos possíveis.

\begin{lstlisting}[language=C++]
int main() {
    int n;
    cout << "Numero de vertices: ";
    cin >> n;

    listarSubgrafos(n);
}
\end{lstlisting}

\end{document}
